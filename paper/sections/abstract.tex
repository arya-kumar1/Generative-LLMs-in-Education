\begin{abstract}
As generative AI continues to integrate into digital learning environments, questions have emerged about how effectively online information sources—particularly Wikipedia—support meaningful learning. While Wikipedia is widely used by students, the educational quality of its articles varies significantly. 

This project investigates how large language models (LLMs) can be trained to evaluate and classify Wikipedia articles based on their pedagogical value. Drawing from research on instructional design, text coherence, and cognitive load theory, the first phase examines which linguistic and structural features make educational materials effective for learning. The second phase applies these insights to train an LLM that diagnoses whether a given Wikipedia article facilitates understanding or promotes misconceptions. 

Using the Wikipedia API and OpenAI’s Apps SDK, the system enables real-time interaction between Wikipedia content and ChatGPT, allowing dynamic analysis and feedback on article quality. By combining educational theory with AI-driven analysis, this project aims to create a tool that not only identifies high-quality learning resources but also informs how AI can enhance open-access educational ecosystems.
\end{abstract}