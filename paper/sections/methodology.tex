\section{Methodology}
\label{sec:methodology}

\subsection{Overview}
Our approach combines educational text analysis with LLM-based evaluation in two tracks.
\textbf{Track A} investigates which linguistic and structural features make Wikipedia
articles effective for learning. \textbf{Track B} develops a ``Wiki Diagnostic Tutor'' that
applies those insights to classify and explain article quality. Together, these tracks
address the research question: \emph{Which Wikipedia articles are good for learning versus
bad for learning, and how can prompt-engineered LLMs identify that difference?}

\subsection{Data Collection}
For \textbf{Track A}, we performed a systematic review of research on instructional
text design, focusing on coherence, scaffolding, and conceptual density. Using these
insights, we defined a set of criteria describing what makes an article effective for
learning (see below). We then compiled a labeled dataset of Wikipedia articles drawn from
topics in mathematics, computer science, and economics that were manually judged as
``good for learning.'' Example articles include:
\begin{itemize}
    \item \href{https://en.wikipedia.org/wiki/Pythagorean_theorem}{Pythagorean Theorem},
    \href{https://en.wikipedia.org/wiki/Golden_ratio}{Golden Ratio},
    \href{https://en.wikipedia.org/wiki/Fibonacci_sequence}{Fibonacci Sequence}
    \item \href{https://en.wikipedia.org/wiki/Eigenvalues_and_eigenvectors}{Eigenvalues and Eigenvectors},
    \href{https://en.wikipedia.org/wiki/Neural_network_(machine_learning)}{Neural Networks},
    \href{https://en.wikipedia.org/wiki/Machine_learning}{Machine Learning}
    \item \href{https://en.wikipedia.org/wiki/2008_financial_crisis}{2008 Financial Crisis},
    \href{https://en.wikipedia.org/wiki/Investment_banking}{Investment Banking}
\end{itemize}
Each article’s revision ID and access date were recorded for reproducibility.

For \textbf{Track B}, we used the Wikipedia API, accessed via the OpenAI Apps SDK
\cite{openaiAppsSDK}, to dynamically retrieve article summaries and content. The prototype
uses prompt engineering to help LLMs reason step-by-step about whether an article
satisfies the educational criteria, rather than simply classifying it directly.

\subsection{Evaluation Criteria}
Articles were judged on ten instructional dimensions derived from cognitive and
instructional design literature:
\begin{enumerate}
    \item \textbf{Engaging Introduction} – presence of historical context or motivation.
    \item \textbf{Real-World Relevance} – clear explanation of why the topic matters.
    \item \textbf{Motivating Examples} – intuitive examples before formal definitions.
    \item \textbf{Visual and Symbolic Support} – inclusion of equations, images, or diagrams.
    \item \textbf{Gradual Progression} – movement from simple to abstract ideas.
    \item \textbf{Balanced Approach} – mix of intuitive and formal reasoning.
    \item \textbf{Multiple Representations} – verbal, visual, and symbolic presentation.
    \item \textbf{Comprehensive but Accessible} – complete coverage without excessive jargon.
    \item \textbf{Multiple Proofs and Perspectives} – variety of explanations or derivations.
    \item \textbf{Applications and Impact} – real-world uses or implications.
\end{enumerate}
These dimensions informed both manual ratings and the prompt templates used for model
evaluation.

\subsection{Data Processing}
We removed markup, standardized section headings, and converted text to plain UTF-8 for
tokenization. Articles were formatted into JSON objects containing title, revision ID, and
text fields. For reproducibility, all preprocessing and classification scripts are available
in our GitHub repository:
\href{https://github.com/arya-kumar1/Generative-LLMs-in-Education}{\texttt{Generative-LLMs-in-Education}}.

\subsection{Analysis Methods}
In \textbf{Track A}, we analyzed relationships between textual features (readability,
sentence length, and structure) and perceived educational quality through descriptive
comparisons and manual rubric scoring.

In \textbf{Track B}, we applied prompt-engineered LLMs using a chain-of-thought reasoning
approach. The model received both the article text and the instructional criteria and was
asked to produce an evaluation explaining whether the article is pedagogically effective,
neutral, or ineffective. Evaluation metrics included precision, recall, and qualitative
interpretation accuracy compared to human-labeled judgments.

\subsection{Integration and Reproducibility}
Outputs from Track A informed prompt phrasing in Track B, ensuring the model aligned with
human-defined learning features. All code, example prompts, and result logs are available at
\href{https://github.com/arya-kumar1/Generative-LLMs-in-Education}{\texttt{github.com/arya-kumar1/Generative-LLMs-in-Education}},
which includes a README detailing dataset structure, API setup, and usage examples.

\subsection{Ethical Considerations}
All data consist of publicly available Wikipedia content accessed under the Wikimedia
\href{https://foundation.wikimedia.org/wiki/Policy:Terms_of_Use}{Terms of Use}.
No personal data were used, and human evaluators remained anonymous. % If you prefer, paste the full section below instead of \input.

\section{Methodology}
\label{sec:methodology}

\subsection*{Overview}
We follow a two–track design. \textbf{Track A} identifies linguistic/structural correlates of educational quality via literature-driven feature engineering and empirical analysis. \textbf{Track B} implements a CoT-enabled “Wiki Diagnostic Tutor” that retrieves Wikipedia content via API and classifies articles with explanations. An additional Week-7 study evaluates the educational suitability of lead summaries using a simple clarity–comprehensiveness–readability rubric.

\subsection{Data Collection}
\textbf{Track A.} We assembled a labeled set of Wikipedia articles rated for educational quality by prior research or expert raters (with revision IDs and access dates). We also compiled a review corpus on instructional quality, writing attitudes, and critical thinking \cite{ekholm2018clarifying, pithers2000critical, fadhly2022efl}. Articles were fetched via the Wikimedia REST API summary and content endpoints \cite{wikimediaREST}, with request metadata logged.

\textbf{Track B.} The prototype connects to the Wikipedia API using the OpenAI Apps SDK tool interface to fetch the page summary (and optionally full text) at inference time \cite{openaiAppsSDK}. 

\textbf{Week-7 Summaries Study.} We executed \texttt{wiki\_summarizer.py} on 5–10 topics (e.g., “AI,” “Machine learning,” “Neural network,” “Overfitting,” “Photosynthesis”) and recorded printed summaries and rubric scores (clarity, comprehensiveness, readability; 1–5 scale), producing per-topic and average scores.

\subsection{Data Processing}
We stripped markup and references, preserved section hierarchy, and tokenized sentences/words. We computed readability (Flesch–Kincaid, Coleman–Liau), hierarchy depth, mean sentences per section, example density (examples per 1{,}000 words), discourse cues (connectives), and lexical diversity. For labels from multiple sources, two expert raters adjudicated disagreements; inter-rater reliability was assessed with Cohen’s~$\kappa$.

\subsection{Analysis Methods}
\textbf{Track A (Feature Discovery).} We tested associations between features and labels using ANOVA/Kruskal–Wallis and logistic regression with FDR correction; we report effect sizes (Cohen’s~$d$, Cliff’s~$\delta$) and domain robustness.

\textbf{Track B (Modeling).} We frame tri-class classification with CoT rationales. The system uses: (i) a Wikipedia retrieval tool; (ii) few-shot CoT prompts emphasizing coherence, scaffolding, example use, and hierarchy; (iii) instruction-tuning on annotated examples. Splits are stratified (70/15/15) at the article level.

\textbf{Evaluation.} Automated metrics include accuracy, macro-precision/recall/F1 with 95\% bootstrap CIs. Human experts blind-rate a 10\% sample for validity and reasoning clarity; we compute Spearman~$\rho$ and $\kappa$. Ablations remove Track-A-inspired cues to quantify contribution deltas.

\subsection{Week-7 Rubric Results (Baseline)}
Average summary score across topics $\approx 4.1$/5 (\emph{high suitability}). The most effective summaries followed a “definition $+$ key ideas/applications” pattern (e.g., AI, Photosynthesis), suggesting lead summaries can serve as early indicators for article-level educational quality.

\subsection{System and Reproducibility}
All scripts (preprocessing, modeling, evaluation) are version-controlled with fixed seeds and an environment lockfile (Python~3.11; \texttt{pandas}, \texttt{numpy}, \texttt{scikit-learn}, \texttt{spacy}, \texttt{textstat}). We log prompts, tool I/O (titles, revision IDs), and outputs. API use complies with Wikimedia Terms of Use; we respect rate limits.

\subsection{Ethical Considerations}
Data are public Wikipedia content accessed under the Wikimedia Terms of Use. We collect minimum necessary metadata, anonymize rater IDs, and release revision IDs for reproducibility.
